%% LaTeX file for Design representation
%% design.tex
%% 
%% Karlsruhe Institute of Technology
%% Version 1.0, 2018-12-13

%% Available page modes: oneside, twoside
%% Available languages: english, ngerman
%% Available modes: draft, final (see README)
\documentclass[twoside, english, draft]{design}

\usepackage{graphicx}
\usepackage{caption}
\usepackage{pdfpages}
\usepackage[export]{adjustbox}

%usepackage{lipsum}


%% ---------------------------------
%% | Information about the thesis  |
%% ---------------------------------

%% Name of the author
\author{PSE Group}

%% Title (and possibly subtitle) of the thesis
\title{
Design}

%% Type of the thesis 
\thesistype{PSE}

%% The advisors are PhD Students or Postdocs
\advisor{M.Sc. Ankush Meshram}
%\begin{document}


%\end{document}
\thispagestyle{empty}

\settitle

%% --------------------------------
%% | Settings for word separation |
%% --------------------------------

%% Describe separation hints here.
%% For more details, see 
%% http://en.wikibooks.org/wiki/LaTeX/Text_Formatting#Hyphenation
\hyphenation{
% me-ta-mo-del
}

%% --------------------------------
%% | Bibliography                 |
%% --------------------------------

%% Use biber instead of BibTeX, see README
\usepackage[citestyle=numeric,style=numeric,backend=biber]{biblatex}
\usepackage{microtype}

\addtolength{\belowcaptionskip}{-10pt}
\setlength{\textfloatsep}{10pt plus 1.0pt minus 2.0pt}
%\addtolength{\abovecaptionskip}{-100pt}
\frenchspacing
%% ====================================
%% ====================================
%% ||                                ||
%% || Beginning of the main document ||
%% ||                                ||
%% ====================================
%% ====================================
\begin{document}
\nocite{*}

%% Set PDF metadata
\setpdf

%% Set the title
\maketitle

%% ----------------
%% |   Abstract   |
%% ----------------

%% The text is included from the following files:
%% - sections/abstract
\thispagestyle{empty}
\begin{abstract}
	\thispagestyle{empty}
\end{abstract}

%% -----------------
%% |   Main part   |
%% -----------------
\thispagestyle{empty}
\newpage
\thispagestyle{empty}
\tableofcontents
\cleardoublepage
\setcounter{page}{1}


\section{Design}\label{sec:intro}
\subsection{Front-End}

\subsubsection{Sequence Diagram}
\includegraphics[max size={\textwidth}{\textheight}]{login1.pdf}
\newpage
\includegraphics[max size={\textwidth}{\textheight}]{filterchain.pdf}
\newpage
\includegraphics[max size={\textwidth}{\textheight}]{option.pdf}

\subsubsection{Activity Diagram}
\newpage

\subsubsection{UI Structure Diagram}
\begin{figure}
	\includegraphics[angle=90,origin=c,max size={\textheight}{\textwidth}]{UI_tree.png}

\end{figure}
\newpage

\subsubsection{Class Diagram}
\includegraphics[max size={\textwidth}{\textheight}]{filterclassdiagram.pdf}
\newpage
\includegraphics[max size={\textwidth}{\textheight}]{frame.pdf}
\newpage
\includegraphics[max size={\textwidth}{\textheight}]{login2.pdf}

\subsection{Back-End}
This subsection deals with the back-end of the ADIN INSPECTOR. How the system deals with client http calls, and how kafka interacts with the system.
An overview of the system can be seen in \autoref{fig:class_back_end}

\subsubsection{Class Diagram}
Next we'll look at each class and method in detail

\begin{itemize}

	\item[•] Config properties file
	      \\The config file is stored alongside the built application .jar file and contains the path to the Kafka installation folder, the user name and password of a mongoDB account with the highest level of access and the name of the database.

	\item[•]Initializer
	      \\Methods:
	      \begin{itemize}
		      \item[-]main
		            \\ parameters: String of arguments from the console
		            \\ returns: void
		            \\ App entry point.
		            \\ We load the config.properties life and use the path provided to start the zookeper, kafka and mongodb services
	      \end{itemize}


	\item[•]MongoConsumer
	      \\The Mongo Consumer, as the name implies, consumes all messages from all topics in the Kafka messaging system. Once a message is found it is passed along to the Mongo Client for further processing.
	      \\Variables
	      \begin{itemize}
		      \item[-]clientMediator
		            \\Type : MongoClientMediator
		            \\ An instance of the Mongo Client Mediator, created with the credentials from the config file.
	      \end{itemize}
	      Methods
	      \begin{itemize}
		      \item[-]MongoConsumer constructor
		            \\parameters: user name and password of a mongoDB account with the highest level of access.
		            \\returns: void
		            \\ Initializes the MongoClient variable and calls listenForRecords();
		      \item[-]getAllTopics
		            \\parameters: none
		            \\returns: an array of strings containing all the available kafka records
		            \\Asks the kafka server service which topics exists.
		      \item[-]listenForRecords
		            \\parameters: none
		            \\returns: void
		            \\This Method first calls getAllTopics and uses the array of topics to poll the kafka server for new messages.
		            \\If new messages are found then the messages are passed to the Mongo Mediator for adding them to the Database.
		            \\If no new messages are found for a topic notify the Mongo Mediator that the collection tied to the topic is ready for pre-processing.
	      \end{itemize}


	\item[•]MongoClientMediator
	      This object serves as a nexus between the users who want to get data out of the database and the consumer, and dataProcessor who want to add data into the database.
	      \\Variables
	      \begin{itemize}
		      \item[-] client
		            \\type: MongoClient
		            \\ An instance of the Mongo Client from the official java API.
	      \end{itemize}
	      Methods
	      \begin{itemize}
		      \item[-]MongoClientMediator constructor
		            \\parameters: Username and password
		            \\returns: void
		            \\Initializes the client variable, throws an error if the user is not found.
		      \item[-]addRecordToCollection
		            \\parameters: String representation of a record in json format
		            \\String name of the collection it should be added to.
		            \\returns: void
		            \\Converts the json string into a java object, then to a bson document and uses the mongoAPI to insert it into the database.

		      \item[-]addRecordsToCollection
		            \\parameters: String Array of records to be added to a collection
		            \\String name of the collection it should be added to.
		            \\returns: void
		            \\for each oneof the members of the array call addRecordToCollection

		      \item[-]ProcessCollection QUESTIONS FOR ANKUSH
		            \\parameters: String, name of a collection
		            \\returns: void

		      \item[-]getCollection
		            \\parameters: String, name of a collection
		            \\returns: String array containing all entries of the collection

		      \item[-]getStartRecord
		            \\parameters: String, name of a collection
		            \\returns: the first entry of the collection as a String.

		      \item[-]getEndRecord
		            \\parameters: String, name of a collection
		            \\returns: the last entry of the collection as a String.

		      \item[-]getCollectionSize
		            \\parameters: String, name of a collection
		            \\returns: the number of entries in the collectoin as int

		      \item[-]getCollectionInRange
		            \\parameters: String, key of the parameter used for filtering
		            \\String start and end ranges for the filtering
		            \\returns: String array containing all entries of the collection within that range
		            \\this Method is very general to allow for flexibility.For example by letting the key be, SourceIPaddresses, or a timeStamp.

		      \item[-]getCollectionInRange
		            \\parameters: String, key of the parameter used for filtering
		            \\String start and end ranges for the filtering
		            \\returns: number of elements matching the range as int
		            \\this Method is very general to allow for flexibility.For example by letting the key be, SourceIPaddresses, or a timeStamp.

	      \end{itemize}

	\item[•]Record
	      \\Every message that comes from kafka and needs to be added to the database has it's own Record class that inherit from this one.
	      \\Every single class that inherits needs to be able to, using reflection, convert itself into a Bson Document where every variable is a key Value pair of the name of the variable and it's associated value.
	      \\Variables
	      \begin{itemize}
		      \item[-] id
		            \\type: String
	      \end{itemize}
	      Methods
	      \begin{itemize}
		      \item[-]getAsDocument()
		            \\parameters: none
		            \\returns: A Document, containing every variable of any class inheriting from this one.
		            \\This function checks for every variable, gets it's name and value as a string and adds it to the document that it eventually returns.
	      \end{itemize}

	\item[•]PacketRecord
	      \\Inheriting from Record, this class contains the variables that match the json string obtained from kafka.
	      \\Variables
	      \begin{itemize}
		      \item[-] id
		            \\type: String
		            \\this id is used for determining the ordering when saving to mongoDB, it's the offset of the message in the kafka messaging queue. inherited from Record
		      \item[-] client
		            \\type: String
		      \item[-] L2Protocol
		            \\type: String
		      \item[-] SourceMACAddress
		            \\type: String
		      \item[-] L4Protocol
		            \\type: String
		      \item[-] SourceIPAddress
		            \\type: String
		      \item[-] PacketSummary
		            \\type: String
		      \item[-] DestinationIPAddress
		            \\type: String
		      \item[-] Timestamp
		            \\type: String
		      \item[-] DestinationPort
		            \\type: String
		      \item[-] SourcePort
		            \\type: String
		      \item[-] DestinationMACAddress
		            \\type: String

	      \end{itemize}

	      Methods
	      \begin{itemize}
		      \item[-]getters / setters
		            \\parameters: none
		            \\returns: variable type
		            \\Each variable has it's getters and setter methods.
	      \end{itemize}

	\item[•]AlarmRecord
	      \\Variables
	      \begin{itemize}
		      \item[-] method
		            \\ An instance of the Mongo Client from the java API
	      \end{itemize}
	      Methods
	      \begin{itemize}
		      \item[-]getters / setters
		            \\parameters: none
		            \\returns: variable type
		            \\Each variable has it's getters and setter methods.
	      \end{itemize}

	\item[•]AggregatedRecord
	      \\Variables
	      \begin{itemize}
		      \item[-] method
		            \\ An instance of the Mongo Client from the java API
	      \end{itemize}
	      Methods
	      \begin{itemize}
		      \item[-]getters / setters
		            \\parameters: none
		            \\returns: variable type
		            \\Each variable has it's getters and setter methods.
	      \end{itemize}



\end{itemize}

\begin{figure}
	\includegraphics[angle=90,origin=c,max size={\textwidth}{\textheight}]{class_back_end.png}
	\caption{This is the class diagram for the whole back-end system}
	\label{fig:class_back_end}
\end{figure}
\newpage

\subsubsection{Sequence Diagram}
\includegraphics[max size={\textwidth}{\textheight}]{sequence_server_init.png}
\subsubsection{Activity Diagram}


\end{document}
