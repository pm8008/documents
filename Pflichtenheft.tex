%% LaTeX2e class for seminar theses
%% seminar.tex
%% 
%% Karlsruhe Institute of Technology
%% Institute for Program Structures and Data Organization
%% Chair for Software Design and Quality (SDQ)
%%
%% Dr.-Ing. Erik Burger
%% burger@kit.edu
%%
%% Version 1.0.1, 2018-04-16

%% Available page modes: oneside, twoside
%% Available languages: english, ngerman
%% Available modes: draft, final (see README)
\documentclass[twoside, english, draft]{Pflichtenheft}

%% ---------------------------------
%% | Information about the thesis  |
%% ---------------------------------

%% Name of the author
\author{PSE GRUPPE}

%% Title (and possibly subtitle) of the thesis
\title{Anomaly Detection in Industrial Networks}

%% Type of the thesis 
% \thesistype{Seminar Thesis}

%% Change the institute here, ``IPD'' is default
 \myinstitute{ KOMPETENZZENTRUM FÜR ANGEWANDTE SICHERHEITSTECHNOLOGIE }

%% The advisors are PhD Students or Postdocs
\advisor{M.Sc. Ankush Meshram}

\settitle

%% --------------------------------
%% | Settings for word separation |
%% --------------------------------

%% Describe separation hints here.
%% For more details, see 
%% http://en.wikibooks.org/wiki/LaTeX/Text_Formatting#Hyphenation
\hyphenation{
% me-ta-mo-del
}

%% --------------------------------
%% | Bibliography                 |
%% --------------------------------

%% Use biber instead of BibTeX, see README
\usepackage[citestyle=numeric,style=numeric,backend=biber]{biblatex}
\addbibresource{pflichtenheft.bib}

%% ====================================
%% ====================================
%% ||                                ||
%% || Beginning of the main document ||
%% ||                                ||
%% ====================================
%% ====================================
\begin{document}

%% Set PDF metadata
\setpdf

%% Set the title
\maketitle

%% ----------------
%% |   Abstract   |
%% ----------------
 This Document outlines the requirements (both functional and non-functional), environment, target audience, and use cases of the software described below.
%% The text is included from the following files:
%% - sections/abstract

%\begin{abstract}
%\%input{sections/abstract.tex}
%\end{abstract}

%% -----------------
%% |   Main part   |
%% -----------------

\thispagestyle{empty}
\newpage
\thispagestyle{empty}
\tableofcontents
\cleardoublepage
\setcounter{page}{1}


\section{Purpose}\label{sec:intro}
The goal of this project is to create a software visualization tool for industrial network traffic to simplify the analysis of anomalous behaviour, both in realtime and from captured stored data.
\\
This software is part of the ADIN framework and is referered to as the "ADIN Inspector".
\\
One component to achieve this goal is a web interface built with modularity in mind so as to make it easily extendable.
\\
The Web view is able to display a series of diagrams and charts to easily identify the behaviour of the network.
Within the Web view the user has the ability to zoom, select, highlight, and filter out data to better understand the aforementioned behaviour in different OSI layers, as well as visualize the flow rate between network nodes.
\\
To support this Web view a back-end messaging solution is needed. This allows the user to easily switch between multiple streams of captured data.
\\
\section{Overall Description}
\section{Interfaces}
\section{Functional Requirements}
\section{Data Requirements}
\section{Non-Functional Requirements}
\section{Essential Testcases}
\section{Software Modelling}
\subsection{Scenario}
\subsection{Use cases}
\subsubsection{Interactivity}


Visual analytics methods combine interactive visualisations with automated analysis
techniques. This allows the user to decide e.g. which part
of the data he or she wants to explore in more detail.

 A basic principle for visual data exploration was introduced by Shneiderman (1997) by what he called the “The
Visual Information Seeking Mantra: 

Overview first, zoom and filter, then details-ondemand”.
This lets the data analyst define to a certain level what he or she wants
to see and visualise. 

Similar to this, Bertin (1983) specified three “levels of reading,”:
The elementary level (allowing the analyst to look at the information about a
single data record), the intermediate level (showing summarised information about a group of data records), and the global level (providing an overview of all data elements).

\subsection{Object Modelling}
\subsection{Dynamic Modelling}
\subsection{User Interface}
\subsection{Glossary}

%% |   Bibliography   |
%% --------------------

%% Add entry to the table of contents for the bibliography
\printbibliography[heading=bibintoc]

\end{document}
