%% LaTeX2e class for seminar theses
%% seminar.tex
%% 
%% Karlsruhe Institute of Technology
%% Institute for Program Structures and Data Organization
%% Chair for Software Design and Quality (SDQ)
%%
%% Version 1.0.1, 2018-04-16

%% Available page modes: oneside, twoside
%% Available languages: english, ngerman
%% Available modes: draft, final (see README)
\documentclass[twoside, english, draft]{Pflichtenheft}

%% ---------------------------------
%% | Information about the thesis  |
%% ---------------------------------

%% Name of the author
\author{PSE GRUPPE}

%% Title (and possibly subtitle) of the thesis
\title{Anomaly Detection in Industrial Networks}

%% Type of the thesis 
% \thesistype{Seminar Thesis}

%% Change the institute here, ``IPD'' is default
 \myinstitute{ KOMPETENZZENTRUM FÜR ANGEWANDTE SICHERHEITSTECHNOLOGIE }

%% The advisors are PhD Students or Postdocs
\advisor{M.Sc. Ankush Meshram}

\settitle

%% --------------------------------
%% | Settings for word separation |
%% --------------------------------

%% Describe separation hints here.
%% For more details, see 
%% http://en.wikibooks.org/wiki/LaTeX/Text_Formatting#Hyphenation
\hyphenation{
% me-ta-mo-del
}

%% --------------------------------
%% | Bibliography                 |
%% --------------------------------

%% Use biber instead of BibTeX, see README
\usepackage[citestyle=numeric,style=numeric,backend=biber]{biblatex}
\addbibresource{pflichtenheft.bib}

\usepackage[nonumberlist]{glossaries}
\makeglossaries

\newglossaryentry{data point}
{
	name=data point,
	plural=data points,
	description={A tuple of values or key-value pairs},
}

\newglossaryentry{data source}
{
	name=data source,
	plural=data sources,
	description={A data source is a service that provides a network interface that can be accessed by the GUI and provides one or more data streams}
}

\newglossaryentry{data stream}
{
	name=data stream,
	plural=data streams,
	description={A sequence of data points. A data stream may grow dynamically, in which case it will provide a kind of "read next" method}
}

\newglossaryentry{diagram}
{
	name=diagram,
	plural=diagrams,
	description={A graphical representation of \glspl{data point} as per a certain \gls{diagram type}. Each data point is represented by a symbol whose position, size, color, shape etc. represent values of the data point. The diagram has labeled axes and, for other properties of the symbol, a legend that allow the user to determine the values represented by each symbol}
}

\newglossaryentry{diagram container}
{
	name=diagram container,
	plural=diagram containers,
	description={The area of the GUI within which the diagram(s) are displayed}
}

\newglossaryentry{diagram type}
{
	name=diagram type,
	plural=diagram types,
	description={A diagram type is a specific style or method of drawing a diagram and visualizing its data points}
}

\newglossaryentry{offline}
{
	name=offline,
	description={a mode in which data is read from the database. Also: data that is read from the database}
}


\newglossaryentry{online}
{
	name=online,
	description={a mode in which data is read from a streaming service or messenger service. Also: data that is read from such a service}
}


\newglossaryentry{role}
{
	name=role,
	plural=roles,
	description={A security role is a list of data sources and diagram types that a user who has this role is allowed to access.}
}


%% ====================================
%% ====================================
%% ||                                ||
%% || Beginning of the main document ||
%% ||                                ||
%% ====================================
%% ====================================
\begin{document}

%% Set PDF metadata
\setpdf

%% Set the title
\maketitle

%% ----------------
%% |   Abstract   |
%% ----------------
 This Document outlines the requirements (both functional and non-functional), environment, target audience, and use cases of the software described below.
%% The text is included from the following files:
%% - sections/abstract

%\begin{abstract}
%\%input{sections/abstract.tex}
%\end{abstract}

\hfill

\begin{center}
    \large{Version 0.5}
\end{center}


%% -----------------
%% |   Main part   |
%% -----------------

\thispagestyle{empty}
\newpage
\thispagestyle{empty}
\tableofcontents
\cleardoublepage
\setcounter{page}{1}


\section{Purpose}\label{sec:intro}
The goal of this project is to create a software visualization tool for industrial network traffic to simplify the analysis of anomalous behaviour, both in realtime and from captured stored data.
\\
This software is part of the ADIN framework and is referered to as the "ADIN Inspector".
\\
One component to achieve this goal is a web interface built with modularity in mind so as to make it easily extendable.
\\
The Web view is able to display a series of diagrams and charts to easily identify the behaviour of the network.
Within the Web view the user has the ability to zoom, select, highlight, and filter out data to better understand the aforementioned behaviour in different OSI layers, as well as visualize the flow rate between network nodes.
\\
To support this Web view a back-end messaging solution is needed. This allows the user to easily switch between multiple streams of captured data.
\\
\section{Overall Description}
Visual Analytics is utilizing graphics to enhance human cognition to understand a problem better or find 'a needle in the haystack' within a huge amount of data. Industrial Network Security aims to understand the communication network traffic generated in an  industrial production system. Analyzing the traffic generated by underlying industrial protocols is the primary step. Real-time visualization of analyzed network data will help the end-user to understand the system's communication behavior and changes within it more clearly. Deviations or incidents can be detected visually as they are occurring or already persisting.
\section{Interfaces}


	The environment of the project is a modern browser with LAN access. The project is OS-agnostic. Therefore the underlying Operating system is irrelevant on the users end.


\subsection{Software}
\begin{itemize}
\item{Client}
\begin{itemize}
	www-browser of the latest generation.
\end{itemize}
	
\item{Server:}
\begin{itemize}
	\item{Java}
	\item{Kafka messaging framework}
	\item{MongoDB database}
\end{itemize}
\end{itemize}
	
\subsection{Hardware}
\begin{itemize}
\item{Client:}
\begin{itemize}
	System capable of network connectivity
\end{itemize}
	
\item{Server:}
\begin{itemize}
	\item{Network capable system}
	\item{System capable of running all the server-software}
	\item{System with adequate storage}
\end{itemize}

\end{itemize}

\section{Functional Requirements}
\subsection{Must Have}

\begin{itemize}
\item{When opening the Web view the user has to be greeted by a login screen.}

\item{There needs to be at least two levels of access for different account types.
Level of access is defined as: the specific set of data streams the user is able to view and select to be analyzed.}
\item{Once logged in an user has to be able to select a data stream to be visualized.}
\item{The user has to have the ability to select multiple diagrams to visualize the selected data stream.}
\item{The user can use at least this diagram types:}
\begin{itemize}
\item{A timeline plot}
\item{A scatter plot}
\item{A Network diagram}
\end{itemize}
\item{The user is able to dynamically change which component of the data is used in both the X and Y axis of the diagram.}
\item{The user should be able to add new diagrams to the GUI and configure them ( i.e. setting diagram type and axis) both at creation and at a later time.
}
\item{The GUI has to support a minimum of 4 different diagrams at once.}
\item{Each diagram should be able to be maximised to take on the full size of the diagrams container.}
\item{Each drawn diagram can be connected to a different data stream.}
\item{The amount of data can be limited via a slider, effectively setting a limited time window, to which all diagram must update to.}
\item{There needs to be an auto scroll function (play button) which automatically scrolls through the selected time window and whose speed is adjustable.}
\item{The user has to be able to log out of the system.}

\end{itemize}

\subsection{Should Have}
\begin{itemize}
\item{The GUI should be able to support an undeterminate number of diagrams and scrollbar.??}
\end{itemize}
\subsection{Nice To Have}
\begin{itemize}
\item{Within the slider the user is able to scroll through the timeline and the diagrams need to react in real-time.
}
\end{itemize}

\section{Data Requirements}
\section{Non-Functional Requirements}

\begin{description}
  \item[NF100]
  The web UI should be able to visualize the network topology and update the visualization based on real time data streams.
  
  \item[NF110]
  The rendering latency should be no longer than 2 seconds.

  \item[NF120]
  The web UI should be view-able on modern web browsers.

  \item[NF200]
  The framework should be able to handle data streamed from at least $\mathit{N}$ sensors.

  \item[NF210]
  The framework should be able to save data collected in a specified period of time.

  \item[NF300]
  The GUI should be easily extendable with more diagram types.

  \item[NF301]
  The GUI should be easily extendable with more filters and aggregations.

  \item[NF302]
  The GUI should be easily extendable with more data types for input data.

  \item[NF400]
  The system needs to provide access control that authenticates users by using user name and password.

  \item[NF402]
  Each users has one or more security \glspl{role}.

  \item[NF404]
  Multiple users can have the same role.

  \item[NF406]
  A role provides authorization to access certain \glspl{data source} and certain \glspl{diagram type}.

  
\end{description}

\section{Essential Test cases}

\begin{description}

  \item[T000] Template Test case
\begin{description}
    \item[Precondition]
	Stuff that happens before
    \item[Action]
    Stuff the user does
    \item[Reaction]
	Stuff the software does
\end{description}

  \item[T100] Successful login
\begin{description}
    \item[Precondition]
    Open browser window.
    \item[Action]
    The user enters the URL for the ADIN web server in the address bar and presses Enter.
    \item[Reaction]
    The browser loads the ADIN website and shows the login screen.

    \item[Precondition]
    ADIN website had been opened and show the login screen.
    \item[Action]
    The user enters a valid username and the corresponding password.
    \item[Reaction]
    Successfully logged in. The browser loads the ADIN main screen with one empty diagram.
\end{description}

  \item[T102] Failed login (wrong username or wrong password)
\begin{description}
    \item[Precondition]
    Open browser window.
    \item[Action]
    The user enters the URL for the ADIN web server in the address bar and presses Enter.
    \item[Reaction]
    The browser loads the ADIN website and shows the login screen.

    \item[Precondition]
    ADIN website has been opened and shows the login screen.
    \item[Action]
    The user enters either a valid username and a incorrect password or a non existing username and an arbitrary password.
    \item[Reaction]
    The login screen shows an error message that the username or password is incorrect and the entry field for the password is cleared.
\end{description}

  \item[T200] Open second diagram.
\begin{description}
    \item[Precondition]
    The browser has been logged in to ADIN and shows the main screen with one empty diagram.
    \item[Action]
    The user presses the "New Diagram" button on the top right side of the screen three times.
    \item[Reaction]
    The browser opens a modal with diagram settings
    \item[Action]
    The user presses the create button at the bottom left
    \item[Reaction]
    The browser opens a second diagram, splitting the diagram panel in to two.

\end{description}


  \item[T202] Open multiple diagrams
\begin{description}
    \item[Precondition]
    The browser has been logged in to ADIN and [T200] has been passed.
    \item[Action]
    The user repeats [T200] two more times.
    \item[Reaction]
    The diagrams grid now displays four diagrams in a 2 x 2 formation.

\end{description}

  \item[T300] Filtering
\begin{description}
    \item[Precondition]
	The browser has been logged in to ADIN and shows at least one diagram
    \item[Action]
    The user enables a filter from the global filters section.
    \item[Reaction]
	The diagram now only shows relevant data
	\item[Action]
    The user disables the same filter.
    \item[Reaction]
	The diagram shows original data again
\end{description}

  \item[T301] Filter chaining
\begin{description}
    \item[Precondition]
	Logged in to ADIN and at least one diagram and one global filter is active
    \item[Action]
    The user enables another filter from the global filters section.
    \item[Reaction]
	The diagram now only shows relevant data
\end{description}

  \item[T400] Full screen a diagram
\begin{description}
    \item[Precondition]
	Logged in to ADIN and shows at least two diagrams
    \item[Action]
    The user presses the full screen button on the top right corner of the diagram
    \item[Reaction]
	The diagrams window is maximized to the diagram grid of the web page.
\end{description}

  \item[T500] Full screen and exit Full screen
\begin{description}
    \item[Precondition]
	Logged in to ADIN and shows at least two diagrams
    \item[Action]
    The user presses the full screen button on the top right corner of the diagram
    \item[Reaction]
	The diagrams window is maximized to the diagram grid of the web page.
    \item[Action]
    The user presses the exit full screen button on the top right corner of the diagram
    \item[Reaction]
	The diagram grid of the web page is restored.
\end{description}

  \item[T600] Full screen and exit Full screen
\begin{description}
    \item[Precondition]
	Logged in to ADIN and shows at least two diagrams
    \item[Action]
    The user presses the full screen button on the top right corner of the diagram
    \item[Reaction]
	The diagrams window is maximized to the diagram grid of the web page.
    \item[Action]
    The user presses the exit full screen button on the top right corner of the diagram
    \item[Reaction]
	The diagram grid of the web page is restored.
\end{description}

  \item[T700] Play button basic functionality
\begin{description}
    \item[Precondition]
	Logged in to ADIN and shows at exactly one diagram.
    \item[Action]
    The user presses the play button at the bottom left of the web page.
    \item[Reaction]
	The play button turns into a stop button.
    \item[Reaction]
	The diagram updates according to the time shown at the play head.
    \item[Action]
	The user presses the stop button.
    \item[Reaction]
	The diagram remains static with the currently displayed data.
\end{description}

  \item[T710] Time window(s)
\begin{description}
    \item[Precondition]
	Logged in to ADIN with one diagram. The play head is in its default position all the way to the left
    \item[Action]
	User drags the left diagram time window key-frame to the right.
    \item[Reaction]
	The timestamp above the keyframe updates according to it's position
    \item[Action]
	User drags the right diagram time window key-frame to the right.
    \item[Reaction]
	The timestamp above the keyframe updates according to it's position
\end{description}

  \item[T711] Multiple Time windows
\begin{description}
    \item[Precondition]
	Logged in to ADIN
	At least 2 diagram windows open
    \item[Action]
    User clicks on one of the diagrams
    \item[Reaction]
	Corresponding keyframes highlight in the timeline and display their timestamp on top of themselves
	\item[Reaction]
	All NOT corresponding keyframes gray out and hide their timestamps
	\item[Action]
	User now clicks on a different diagram
    \item[Reaction]
	Previously highlighted keyframes gray out and keyframes corresponding to selected diagram highlight and display their timestamp
	\item[Reaction]
	All NOT corresponding keyframes gray out and hide their timestamps
\end{description}

  \item[T712] Time Window dragging restriction
\begin{description}
    \item[Precondition]
	Logged in to ADIN
	At least 2 diagram windows open
	Test case [T711] passed
	One diagram has been selected
    \item[Action]
    User tries to drag a not highlighted keyframe from the timeline
    \item[Reaction]
	Unable to drag and drop not highlighted keyframes
\end{description}


\end{description}

\section{Software Modeling}



\subsection{GUI}
The basic data structure needed for graphs are a given set of nodes and a given set of edges, as they are often drawn
as node-link diagrams. In the postal data set, nodes could represent the origins and destinations of postal flows. Edges represent the flows between the respective origins and destinations.In intelligence analysis, investigators use semantic graphs to organize concepts and relationships as
graph nodes and links in hopes of discovering key trends, patterns, and insights.”


A key issue in graph visualization is the size of the graph, i.e. the size of the data to visualize. With a growing amount of data to display, graphs can become too complex and overburdening for the analyst's cognitive capacity. It thus becomes difficult for the user to conduct significant analysis. Because of the issues described above, research often focuses on ways to solve the problems of visual clutter, e.g. by aggregation or clustering techniques, which is also one of the main topics of cartographic generalization.

%%For example, Cui et al. (2008) investigate node-link diagrams for network visualisation.
%Their study focuses on the problem of visual cluttering in graph visualisations,
%as this is one of the main issues in the representation of relationships among
%large data (Herman, Melancon, and Marshall, 2000). They introduce a framework
%for geometry-based edge clustering to group edges into bundles and hence reduce
%visual cluttering caused by the crossing of the high number of edges (see figure 2.1).

\begin{figure}
\centering
\includegraphics{node.jpg}
  \caption[ height=4cm, width=4cm]{Node-Link-Graph}
\end{figure}

\subsection{Scenario}

\begin{description}
\item[S100:] An operator wants to check manually/visually whether network nodes appeared or disappeared over the last day
\begin{itemize}
\item{the operator opens the web page}
\item{the operator selects the database as data source}
\item{the operator selects a time-line-based \gls{diagram type}}
\item{the operator selects node addresses as the data to be displayed}
\item{the operator moves to or selects the last 24 hours as the range of data to display}
\item{the operator closes the web page}
  \end{itemize}



\item[S200:] A security analyst wants to look at the current flow rates between network nodes to see whether they change / there are trends
\begin{itemize}
\item{the analyst opens the web page}
\item{the analyst selects a source of live data}
\item{the analyst selects an appropriate visualization type}
\item{the analyst selects node addresses as the independent variable}
\item{the analyst selects flow rates as the data to be displayed}
\end{itemize}



\item[S300:] A security analyst wants to examine a specific point of data
\\
Precondition: the analyst has already selected the relevant dataset and visualization type
\begin{itemize}
\item{the analyst selects a data point by right-clicking}
\item{the GUI displays a small pop-up window with all the data of this data point}
\item{the analyst right-clicks one of the attibutes in the pop-up window and selects "Display all corresponding types"}
\item{the GUI marks all data points that have the same value in this attribute}
\end{itemize}



\item[S400:] The user wants to look at alarms/notifications
\begin{itemize}
\item{the user opens the web page}
\item{the user selects the database as data source}
\item{the user selects the data stream from the relevant dissector}
\item{the GUI diplays the notifications along a timeline, according to the time of the event}
\item{the user right-clicks on the x-axis and selects "use record number"}
\item{the GUI diplays the notifications along a timeline adjacently}
\end{itemize}



\item[S500:] The user wants to look at normal data together with alarms/notifications
\\
Precondition: Scenario /S100/ apart from closing the web page
\begin{itemize}
\item{the user selects menu "data", entry "sources"}
\item{the GUI displays a list of all known data sources with a checkbox in front of each}
\item{the user selects the checkboxes for the data sources they want to examine}
\item{the GUI displays data from all these data sources within the currently active visualization}
\end{itemize}


\end{description}

\subsection{Use cases}
\subsubsection{Interactivity}


Visual analytics methods combine interactive visualisations with automated analysis
techniques. This allows the user to decide e.g. which part
of the data he or she wants to explore in more detail.

 A basic principle for visual data exploration was introduced by Shneiderman (1997) by what he called the “The
Visual Information Seeking Mantra: 

Overview first, zoom and filter, then details-ondemand”.
This lets the data analyst define to a certain level what he or she wants
to see and visualise. 

Similar to this, Bertin (1983) specified three “levels of reading,”:
The elementary level (allowing the analyst to look at the information about a
single data record), the intermediate level (showing summarised information about a group of data records), and the global level (providing an overview of all data elements).

\subsection{Quality Requirements}

\begin{tabular}{l*{3}{c}r}
Product Quality              & really good & good & normal & not relevant  \\
\hline
\textbf{Functionality} &  &  &  &    \\
Appropriateness & & x & & \\
Accuracy & & x & & \\
Interoperability & & x & & \\
Regularity & & x & & \\
Security & x&  & & \\
\textbf{Reliability} &  &  &  &    \\
Maturity & &  & x& \\
Error tolerance & &  &x & \\
Recoverability& &  & x & \\
\textbf{Usability} &  &  &  &    \\
Understandability & & x & & \\
Learnability & & x & & \\
Usability & x &  & & \\

\textbf{Efficiency} &  &  &  &    \\
Time behaviour & & x & & \\
Consumption behaviour & & x & & \\
\end{tabular}

\subsection{Object Modelling}
\subsection{Dynamic Modelling}
\subsection{User Interface}

\section{Glossary}
\printglossary[title=,toctitle=]


%% |   Bibliography   |
%% --------------------

%% Add entry to the table of contents for the bibliography
\printbibliography[heading=bibintoc]

\end{document}
